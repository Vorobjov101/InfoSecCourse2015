\documentclass[a4paper,12pt]{article}
\usepackage[utf8]{inputenc}
\usepackage[russian]{babel}

\title{Лабораторные работы №1-3 \\ Курс: Защита информации}
\author{Воробьев Олег}

\ifx\pdfoutput\undefined
\usepackage{graphicx}
\else
\usepackage[pdftex]{graphicx}
\fi

\begin{document}
\maketitle
\clearpage
\tableofcontents
\clearpage

\part[1]{Cистема верстки TEX и расширения LATEX}

Файл .tex представляет из себя обычный текстовый файл содержащий макрокоманды текстовой разметки.

\section[1]{Создание минимального файла .tex в простом текстовом редакторе преамбула, тело документа}

Создаем в блокноте простой файл, включкющий в себя минимум строк(Рис 1).


\begin{figure}[h!]
\centering
\includegraphics[width=0.5\textwidth]{scr1}
\caption{Простейший tex документ.}
\end{figure}

Здесь указывается тип документа, а так же выводится одна простая строка.

\section[2]{Компиляция в командной строке – latex, xdvi, pdflatex}

Файлы latex оегко могут быть скомпилированы прямо из командной строки. В системе эти файлы хранятся в формате, не предназначенном для чтения, поэтому требуется преобразовать их в читаемый вид.
Так компилируется latex файл(Рис 2).
\begin{figure}
	\center
	\includegraphics[width = 6in]{scr2}
	\caption{Компиляция в консоли.}
\end{figure}

Так же tex файла можно пеобразовать в PDF файлы с помошью команды PDFLATEX(Рис 3).
\begin{figure}
	\center
	\includegraphics[width = 6in]{scr4}
	\caption{Компиляция pdf в консоли.}
\end{figure}


\section[3]{Оболочка TexMaker, Быстрый старт, Быстрая сборка}

Вместо texmaker был установлен не менее удобный редактор texstudio. Texstudio - это редактор текста поддерживающий язык разметки Latex. Он реализует всю функциональность, требующуюся для работы с многостраничными документами. Внешний вид редактора выглядит следующим образом(Рис 4).
\begin{figure}
	\center
	\includegraphics[width =6in,height = 100mm]{scr5}
	\caption{Вид texstudio.}
\end{figure}

В редакторе так же реализованы две функции: быстрый старт и быстрая сборка. Первая - позволяет быстро создать шаблон документа(Рис 5).
\begin{figure}
	\center
	\includegraphics{scr6}
	\caption{Создание шаблона документа.}
\end{figure}

Вторая скомпилировать его и преобразовать в читаемый вид. Процесс схож с компиляцией программы. Имеется возможность задания последовательности действий при быстрой сборке.



\section[4]{Создание титульного листа, нескольких разделов, списка, несложной формулы}

Содание титульного листа в самом простом варианте осуществляется следующим образом.
В преамбуле документа нужно указать название документа и автора.
\begin{verbatim*}
	\title{Лабораторные работы №1-3 \\ Курс: Защита информации}
	\author{Воробьев Олег}	
\end{verbatim*}
Затем, сам заголовок создается командой
\begin{verbatim*}
	\maketitle
\end{verbatim*}

В итоге получаем титульный лист(Рис 6).
\begin{figure}
	\center
	\includegraphics{scr7}
	\caption{Титульный лист.}
\end{figure}

Новый раздел создается командой part
\begin{verbatim*}
	part[1]{Раздел 1}
	part[2]{Раздел 2}
	part[3]{Раздел 3}
\end{verbatim*}
В документе это выглядит следующим образом(Рис 7).
\begin{figure}
	\center
	\includegraphics{scr8}
	\caption{Разделы.}
\end{figure}
Списки в LATEX создаются автоматически с помошью нескольких команд. Перед каждым элементом вседа должна идти команда ITEM,внутри блока begin-end реализуется сам список(Рис 8).
\begin{figure}
	\center
	\includegraphics{scr9}
	\caption{Списки.}
\end{figure}
Процесс написания формул схож  таковым в MATLAB. Вышлядит это примерно так:
\begin{verbatim*}
	f\{x,y}\=\frac{x^2+y^2}{\sqrt{x^3+y^3}}
\end{verbatim*}
\begin{math}
	f(x,y) = \frac{x^2+y^2}{\sqrt{x^3+y^3}}
\end{math}

\section[5]{Понятие классов документов, подключаемых пакетов}

Каждый файл в LATEX начинается с команды documentclass[...]{...}, в фигурных скобках которой задаются параметры оформления стиля документа, а в квадратных — список классовых опций.
Всего в LATEX 5 основных классов документов:
\begin{itemize}
\item {article для статей}
\item {report для книг и статей}
\item {book для книг}
\item {proc для докладов}
\item {letter для оформления деловых писем} . 
\end{itemize} 
Помимо этих основных, есть ещё множество дополнительных.


В LATEX помимо стандартных настроек существует возможность подключения сторонних пакетов со специфическими настройками. ТАкие пакеты раширений подключаются в шапке документа.
\begin{verbatim*}
	usepackage{listings} % предоставляет возможности цитирования кода в тексте с сохранением исходного форматирования.
\end{verbatim*}


\section[6]{Верстка более сложных формул}

Рассмотрим формулу проведения консолидированного платежа.
\begin{verbatim*}
	\frac{P_{0}}{1+n_{0}*r}=\sum_{k=1}^{m}\frac{P_{k}}{1+n_{k}*r}
\end{verbatim*}
\begin{math}
\frac{P_{0}}{1+n_{0}*r}=\sum_{k=1}^{m}\frac{P_{k}}{1+n_{k}*r}
\end{math}

\section[6]{Вывод}

LATEX наиболее популярный набор макросов системы компьютерной вёрстки TEX, который облегчает набор сложных документов. Упрощается и автоматизируется процесс написания текста и подготовки статей. Существует множество пакетов расширения LATEX, позволяющих удобно настраивать документ.
Для работы с latex чаще всего используются специализированняе среды, поддерживающие разметку и выделение кода, поволяющие автоматически компилировать документы, предпросматривать конечный результат и тд.

\part[2]{Система контроля версий GIT}

\section[1]{Получить содержимое репозитория}

Содержимое репозитория можно получить простой командой.
\begin{verbatim*}
git clone git@github.com:Vorobjov101/InfoSecCourse2015.git
\end{verbatim*}

\section[2]{Добавить папку и файл в систему}

Создание папки в репозитории и добавление файла в нее.
\begin{verbatim*}
mkdir test
cd test
echo 11001 >> var
git add --all
\end{verbatim*}

\section[3]{Зафиксировать изменения в локальном репозитории}

\begin{verbatim*}
git commit -a -m "file added"
\end{verbatim*}

\section[4]{Внести изменения в файл и посмотреть различия}

\begin{verbatim*}
echo 11111 >> var
git diff master:./var ./var
\end{verbatim*}

\section[5]{Отменить локальные изменения}

\begin{verbatim*}
git reset HEAD ./var
git checkout ./var
\end{verbatim*}

\section[6]{Внести изменения в файл и посмотреть различия}

\begin{verbatim*}
echo 00010101 >> var
git diff master:./file ./file
\end{verbatim*}

\section[7]{Зафиксировать изменения в локальном репозитории, зафиксировать изменения в центральном репозитории}

\begin{verbatim*}
git commit -a -m "file changed"
git push
\end{verbatim*}

\section[8]{Получить изменения из центрального репозитория}

\begin{verbatim*}
git pull
\end{verbatim*}

\section[9]{Поэкспериментировать с ветками}

\begin{verbatim*}
git branch -n
git checkout master
git merge temp
git branch
\end{verbatim*}

\section[10]{Вывод}
Git система предоставляющая возможность управления версиями файлов и  их распределеное хранение. Git используется во множестве проектов для обеспечения совместной работы над проектом. 

\part[3]{Программа для шифрования и подписи GPG, пакет Gpg4win}

\section[1]{Создать ключевую пару OpenGp}

Kleopatra это графический интерфейс к GnuPG и предназначенных для работы под окружением KDE, портированный на MS Windows(Рис 9).
\begin{figure}
	\center
	\includegraphics{scr10}
	\caption{Новая ключевая пара.}
\end{figure}
Выглядит ключ следующим образом(Рис 10).
\begin{figure}
	\center
	\includegraphics{scr11}
	\caption{Ключ.}
\end{figure}

\section[2]{Поставить ЭЦП на файл}
После подписания файла создается го копия, но с измененным форматом. В конце приписывается .sig(Рис 11).
\begin{figure}
	\center
	\includegraphics{scr12}
	\caption{Файл lec2 vorobuov с подписью}
\end{figure}

\section[3]{Получить чужой сертификат,импортировать его, проверить подпись}
Получив чужой сертификат(Рис 12), можно расшифровывать подписанные этим пользователем файлы(Рис 13).
\begin{figure}
	\center
	\includegraphics{scr13}
	\caption{Полученный сертификат.}
\end{figure}
\begin{figure}
	\center
	\includegraphics{scr14}
	\caption{Расшифровка файла с помощью импортированного сертификата.}
\end{figure}

\section[8]{Работа с консолью}
Все эти операции можно повторить в консоли. Например, создание ключа осуществляется командой:
\begin{verbatim*}
	gpg --gen-key
\end{verbatim*}
При создании нас просят ввести дополнительные параметры. Выглядит это примерно так(Рис 14).
\begin{figure}
	\center
	\includegraphics[width =6inc]{scr15}
	\caption{Создание ключа в консоли.}
\end{figure}
Мы можем уедиться, что он создан с помошью команды(Рис 15).
\begin{verbatim*}
	gpg --list-keys
\end{verbatim*}

\begin{figure}
	\center
	\includegraphics{scr16}
	\caption{Отображение всех зарегестрированных в система ключей.}
\end{figure}

Для импорта и экспорта используются команды --import и --armor соответственно.

\end{document}