 \documentclass{article}
 \usepackage[utf8]{inputenc}
 \usepackage[russian]{babel}
 
 \title{Лабораторные работы №4 \\ Курс: Защита информации}
 \author{Воробьев Олег}
 
 \ifx\pdfoutput\undefined
 \usepackage{graphicx}
 \else
 \usepackage[pdftex]{graphicx}
 \fi
 
 \begin{document}
 \maketitle
 \clearpage
 \tableofcontents
 \clearpage
 
 \section{Работа с утилитой nmap}
 \subsection{Начальные настройки}
	В ходе данной работы будут использоваться две виртуальные машины: одна для сканирования, другая- как цель для сканирования. Для сканирования используется система KAlI Linux с предустановленными утилитами. В качестве объекта для сканирования используется metasploitable. Обе эти системы запускаются в с настройками сети в режиме сетевого моста(Рис 1). 
 \begin{figure}[h!]
 \centering
 \includegraphics[width =\linewidth]{scr1}
 \caption{Настройки сети.}
 \end{figure}
 
	Благодаря таким настройкам, две виртуальные машины могут видеть друг друга в локальной сети. Это обезопасит дальнейшую работу с ними. С помощью команды ifconfig можно удостовериться что машины получили свои ip адреса для дальнейшей работы с ними(Рис 2).
 \begin{figure}
	 \centering
	 \includegraphics[width =\linewidth]{scr2}
	 \includegraphics[width =\linewidth]{scr3}
	 \caption{IP адреса машин.}
 \end{figure}
 \clearpage
 
 \subsection{Поиск активных портов}
 Поиск активных хостов осушествляется с помощью следующей команды.
 \begin{verbatim}
 	nmap -sn 192.168.1.0/24
 \end{verbatim}
 Эта команда просканирует локальную сеть на наличие открытых хостов и покажет все доступные для прозванивания IP адреса(Рис 3). Вот результат.
 \begin{figure}[h!]
 	\centering
 	\includegraphics[width =\linewidth]{scr4}
 	\caption{Список активных хостов.}
 \end{figure}
 
 Последние 2 адреса - это адреса виртуальных машин запущенных на данном компьютере. Адрес 192.168.1.19 это адрес metasploitable к которому мы будем обращаться в течении работы.
 \clearpage
 
 \subsection{Определить открытые порты}
 Поиск открытых портов осушествляется с помощью следующей команды.
 \begin{verbatim}
 nmap -open 192.168.1.19
 \end{verbatim}
 Эта команда просканирует IP адрес на наличие открытых портов и покажет все доступные для прозванивания порты(Рис 4). Вот результат.
 \begin{figure}[h!]
 	\centering
 	\includegraphics[width =\linewidth]{scr5}
 	\caption{Список всех открытых портов по данному IP  адресу.}
 \end{figure}
 
 Стоит отметить что показываются только открытые порты на данном адресе. Для отображения всех портов следует использовать другую команду.
 \begin{verbatim}
 nmap -p 0-65536 192.158.1.19
 \end{verbatim}
 Обработка такого запроса анимает гораздо больше времени, чем предыдущего, так как программа обращается ко всем возможным портам по данному IP адресу.
 \clearpage
 
 \subsection{Определить версии портов}
 Отображение версий портов осуществляется следующей командой.
 \begin{verbatim}
 nmap -sV 192.168.1.19
 \end{verbatim}
 Результат обработки такого запроса представлен на рисунке 5.
 \begin{figure}[h!]
 	\centering
 	\includegraphics[width =\linewidth]{scr6}
 	\caption{Список открытых портов с версиями сервисов.}
 \end{figure}
 \clearpage
 \subsection{Изучить файлы nmap-services, nmap-os-db,nmap-service-probes}
 Файл nmap-services содержит набор наиболее часто используемых портов и сервисов на них. Подключение этого файла к процессу сканирования позволяет ускорить обработку команды, так как опрос будет проводиться не по всем 65536 портам а только потем, которые указаны в файле. Содержимое файла представлено на рисунке 6.
 
 \begin{figure}[h!]
 	\centering
 	\includegraphics[width =\linewidth]{scr7}
 	\caption{Содержимое файла nmap-servives.}
 \end{figure}
 
 Этот файл задействуется с помощью опции -F. Выглядит это примерно так.
 \begin{verbatim}
 nmap -F 192.168.1.19
 \end{verbatim}
 \clearpage
 \begin{figure}[h!]
 	\centering
 	\includegraphics[width =\linewidth]{scr8}
 	\caption{Результат работы nmap c опцией -F.}
 \end{figure}

 Как видно на рисунке 7, результат не на много, но все же быстрее обычного.
 
 Файл nmap-os-db содержит слепки откликов различных операционных систем на запрос nmap. Благодаря этому файлу утилита может распознавать различные операционные системы с которыми она в данный момент работает. 
 \begin{figure}[h!]
 	\centering
 	\includegraphics[width =\linewidth]{scr9}
 	\caption{Содержимое файла nmap-os-db.}
 \end{figure}
 \clearpage
 Для задействования этого файла используется опция -О. Команда выглядит так
 \begin{verbatim}
 nmap -O 192.168.1.19
 \end{verbatim}
 
 В результате работы помимо информации о портах так же выводится информация о опрашиваемой системе.
 \begin{figure}[h!]
 	\centering
 	\includegraphics[width =\linewidth]{scr10}
 	\caption{Дополнительный вывод утилиты.}
 \end{figure}
 
 База данных nmap-service-probes содержит запросы для обращения к различным службам и соответствующие выражения для распознавания и анализа ответов.
  
 \begin{figure}[h!]
 	\centering
 	\includegraphics[width =\linewidth]{scr11}
 	\caption{Содержимое файла nmap-service-probes.}
 \end{figure} 
 \clearpage
 
 \subsection{Добавить новую сигнатуру службы в файл nmap-service-probes}
 
 Для наглядной демонстрации работы файла nmap service probes запустим примитивный TCP сервер на порте 10100. 
 
 Просканировав даный порт с помошью nmap мы увидим что его поле сервиса не заполнено.
  \begin{figure}[h!]
  	\centering
  	\includegraphics[width =\linewidth]{scr12}
  	\caption{Сканирование порта 10100 без изменения файла.}
  \end{figure}
  
  Теперь отредактируем файл nmap service probes. Добавим в него следующие строки:
  \begin{figure}[h!]
  	\centering
  	\includegraphics[width =\linewidth]{scr13}
  	\caption{Внесенные изменения в файл nmap service probes.}
  \end{figure}
 
 
 Если построчно то мы делаем следующее:
 \begin{itemize}
 	\item создаем новый запрос TCP MYPROBE
 	\item прикрепляем запрос к порту 10100
 	\item в случае если получен ответ "this is my server" то система распознает наш сервер 
 \end{itemize}
 
 Теперь запустим утилиту еще раз но уже с обновленным файлом.
 \begin{figure}[h!]
 	\centering
 	\includegraphics[width =\linewidth]{scr14}
 	\caption{Обновленный вывод утилиты.}
 \end{figure}
 Как видно, система распознала наш сервер и вместо сервиса unknown тереь указан tcp.
 
 \subsection{Сохранить вывод утилиты в XML}
 
 Сохранение вывода утилиты в формате xml  осуществляется всего командой:
 \begin{verbatim}
 	nmap -sV -oX -some_output 192.168.1.4
 \end{verbatim}
 
 \subsection{Исследовать работу nmap  с применением WIRESHARK}
 Проанализируем с помошью wireshark каким путем nmap работает с компьютером.
 
  \begin{figure}[h!]
  	\centering
  	\includegraphics[width =\linewidth]{scr15}
  	\caption{Демонстрация работы nmap в Wireshark.}
  \end{figure}
  
  Как видно из скриншота утилита устанавливает TCP соединения со всеми портами и анализирует ответы на эти запросы.
 
 \subsection{Просканировать виртуальную утилиту Metasploitable2 используя db nmap из состава metasploit framework.\\}
 
 Предварительно необходимо включить postgresql и metasploit
 \begin{verbatim}
 	service postgresql start
 	service metasploit start
 	msfconsole
 \end{verbatim}
 \begin{figure}[h!]
 	\centering
 	\includegraphics[width =\linewidth]{scr16}
 	\caption{Подключение.}
 \end{figure}
 
 Затем запускаем команду
  \begin{verbatim}
  db_nmap -v -sV 192.168.124.211
  \end{verbatim}
  Результат сканирования приведен на рисунке ниже.
  
 \subsection{Рассмотрите один скрипт из состава NMAP.\\}
 Рассмотрим скрипт auth-spoof.nse
 \begin{verbatim}
 	local comm = require "comm"
 	local shortport = require "shortport"
 	
 	description = [[
 	Checks for an identd (auth) server which is spoofing its replies.
 	
 	Tests whether an identd (auth) server responds with an answer before
 	we even send the query.  This sort of identd spoofing can be a sign of
 	malware infection, though it can also be used for legitimate privacy
 	reasons.
 	]]
 	
 	---
 	-- @output
 	-- PORT    STATE SERVICE REASON
 	-- 113/tcp open  auth    syn-ack
 	-- |_auth-spoof: Spoofed reply: 0, 0 : USERID : UNIX : OGJdvM
 	
 	author = "Diman Todorov"
 	
 	license = "Same as Nmap--See http://nmap.org/book/man-legal.html"
 	
 	categories = {"malware", "safe"}
 	
 	
 	portrule = shortport.port_or_service(113, "auth")
 	
 	action = function(host, port)
 	local status, owner = comm.get_banner(host, port, {lines=1})
 	
 	if not status then
 	return
 	end
 	
 	return "Spoofed reply: " .. owner
 	end
 \end{verbatim}
  
  Сначала объявляются переменные.
  \begin{verbatim}
  	local comm = require "comm"
  	local shortport = require "shortport"
  \end{verbatim}
  
  Затем соледует описание скрипта.
  \begin{verbatim}
  		description = [[
  		Checks for an identd (auth) server which is spoofing its replies.
  		
  		Tests whether an identd (auth) server responds with an answer before
  		we even send the query.  This sort of identd spoofing can be a sign of
  		malware infection, though it can also be used for legitimate privacy
  		reasons.
  		]]
  \end{verbatim}
  
  Затем идет правило порта. Эта функция возвращет true если ответ совпал с правилом и false если нет.
  \begin{verbatim}
  	portrule = shortport.port_or_service(113, "auth")
  \end{verbatim}
  
  Затем описано действие.
  \begin{verbatim}
  	action = function(host, port)
  	local status, owner = comm.get_banner(host, port, {lines=1})
  	
  	if not status then
  	return
  	end
  \end{verbatim}
  
  Данный скрипт просто проверяет отклик порта, перед началом работы с ним.
  
  \section{Вывод}
  В ходе данной работы были изучены основные приемы работы с nmap, wireshark и metasploit. Nmap является многогранным инструментом для сбора информации о компьютере. Это может стать хорошим подспольем в подготовке атаки или в анализе слабых мест машины.
 \end{document}