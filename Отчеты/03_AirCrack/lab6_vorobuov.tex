\documentclass{article}
\usepackage[utf8]{inputenc}
\usepackage[russian]{babel}
 
\title{Лабораторная работа №6 \\ Курс: Защита информации}
\author{Воробьев Олег}
 
\ifx\pdfoutput\undefined
\usepackage{graphicx}
\else
\usepackage[pdftex]{graphicx}
\fi

\begin{document}
\section{Цель работы}
Изучить основные возможности пакета AIR Crack и принципы взлома WPA/WPA2 PSK и WEP.
\section{Ход работы}
\subsection{Изучение}
\paragraph{Изучить документацию по основным утилитам пакета airmon-ng,airodump-ng,aireplay-ng,aircrack-ng.\\}

Кратко опишем назначение этих утилит.
\begin{enumerate}
	\item airmon-ng Включает wifi адапрет в режим мониторинга. Это позволяет ему перехватывать не только пакеты предназначенные непосредственно ему, но и все доступные пакеты в сети. 
	\item airodump-ng Утилита, предназначенная для захвата сетевых пакетов протокола 802.11 для последующего их использования в последствии
	\item aireplay-ng Утилита, которая генерирует и отсылает трафик. Может использоваться для принудительной деаутентификации пользователей.
	\item aircrack-ng Непосредственно взламывает пароли WEP WPA используя перебор по словарю.
\end{enumerate}

\paragraph{Запустить режим мониторинга на беспроводном интерфейсе}
\paragraph{Запустить утилиту airodump, изучить формат вывода этой утилиты, форматы файлов, которые она может создавать.}
\subsection{Практическое задание}
\paragraph{Запустить режим мониторинга на беспроводном интерфейсе.}
\paragraph{Запустить сбор трафика для получения аутентификационнах сообщений.}
\paragraph{Если аутентификации не происходит, произвести деаутентификацию одного из клиентов.}
\paragraph{Произвести взлом используя словать паролей.}
\section{Вывод}
	
\end{document}