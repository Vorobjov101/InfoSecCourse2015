\documentclass{article}
\usepackage[utf8]{inputenc}
\usepackage[russian]{babel}

\title{Эссе №6 \\ Курс: Защита информации}
\author{Воробьев Олег}

\ifx\pdfoutput\undefined
\usepackage{graphicx}
\else
\usepackage[pdftex]{graphicx}
\fi

\begin{document}
	\maketitle
	\clearpage

В данной статье автор рассматривает вопросы безопасности данных в мобильных приложениях. Автор рассматривает систему Android как наиболее удачный объект исследований. Она привлекательна как очень популярная система для мобильных устройств следовательно, содержащая большой объем информации о данных пользователя. Так же дольшое количество приложений под эту систему делает оюъем данных, контактирующих с ней еще более внушительным.

Например режим ECB. Главная его уязвимость заключается в том, что одинаковые фрагменты данных с помошью этого алгоритма шифруются одинаковым шифром. Дананя система может быть усилена добавлением случауного числа "соли" в функцию, что создаст больше сложности с подбором.

Анализ безопасности проволится на основе собственноручно разработанного инструмента на базе androguard-framework. Основная идея состоит в поиске по исходному коду значений инициализирующих векторов, ключей и т.д. а так же криптоалгоритмов. При помощи этого приложения исследовано 11 748 приложений среди которых 10 327 используют криптографию неправильно, а это больше 85%.


Далее приводится три алгоритма шифрования с различными типами блочного шифрования. Анализируя алгоритмы автор предлагает нам в качестве резюме 6 правил для правильного использования защита информации в Android системах.

\begin{itemize}
\item Не использовать ECB режим при криптографии
\item Не использовать non-randon IV для CBC шифрования
\item Не использовать константные ключи шифрования
\item Не использовать константную соль для шифрования на основе пароля
\item Не использовать менее 1000 итераций для шифрования на основе пароля
\item Не использовать постоянные seed для получения псевдослучайных последовательностей SecureRandom()
\end{itemize}

Правило 1 запрещает использоавть ЕСВ так как эта схема шифрования не предоставляет должных параметров безопасности.
Правило 2 весьма очевидно. Использование динамических колчей шифрования значительно повысит криптоустойчивость системы.
Правило 3 аналогичо правилу 2.
Правило 4 и 5 выведено чисто опытным путем для PBE схем.

Инструмент автора проверяет соблюдение этих правил. Приложения на android не схожи с обычными java приложениями. За их выполнение отвечает виртуальная машина Dalvik, которая имеет мало схожего со стандартной ВМ Java. Такие приложения получают доступ к графическому интерфейсу и подсистемам. При помощи JCA регистрируются cryptographic service
providers (CSP), которые отвечают за большинство алгоритмов. Для использования этих алгоритмов необходимо вызвать метод Cipher.getInstance.

Следует отметить, что по умолчанию выбирается режим шифрования ECB. В статье разобрано подробно, как строились графы потока управления приложения. И показывалось, как в них находились нарушения. Так же проволится анализ нескольких популярных приложений.

В заключении автор еще раз говорит, что 88 процентов протестированных приложений оказались несоответствующими хотя бы одному правилу. Основываясь на выводах с огромного анализа реальных приложений, автор надеется что в дальнейшем безопасность Android систем. 
	
\end{document}